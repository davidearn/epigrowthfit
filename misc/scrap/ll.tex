\documentclass[12pt]{article}

\usepackage[top=1in,bottom=1.5in,left=1in,right=1in]{geometry}
\usepackage{amsmath,bm}
\allowdisplaybreaks
\usepackage[colorlinks=true,allcolors=magenta]{hyperref}
\renewcommand{\vec}[1]{\bm{\mathrm{#1}}}

\begin{document}
\setlength{\parskip}{3mm}
\setlength{\parindent}{7mm}

\section{Likelihood function}

Let $c(t;\vec{\xi})$ denote expected cumulative incidence at time $t$
conditional on model parameters $\vec{\xi}$. Assume interval break
points $t_0 < t_1 < \cdots < t_n$,
and let $\mu_i(\vec{\xi}) = c(t_i;\vec{\xi}) - c(t_{i-1};\vec{\xi})$
for $i = 1,\ldots,n$. Here, $\mu_i(\vec{\xi})$ is the expected number
of cases observed between times $t_{i-1}$ and $t_i$. If the actual number
$X_i$ of cases observed between times $t_{i-1}$ and $t_i$ follows a negative
binomial distribution with mean $\mu_i(\vec{\xi})$ and dispersion $k > 0$,
%
\begin{equation}
  \label{eq:nbinom}
  X_i \sim \mathrm{NegativeBinomial}(\mu_i,k)\,,
\end{equation}
%
and if the set $\{X_i\}_{i=1}^{n}$ is mutually independent, then the
likelihood of model $\vec{\theta} = (\vec{\xi},k)$ given data
$\vec{x} = \{x_i\}_{i=1}^{n}$ is
%
\begin{equation}
  \label{eq:likelihood}
  \mathcal{L}(\vec{\theta}|\vec{x}) = \prod_{i=1}^{n} f_i(x_i;\vec{\theta})\,,
\end{equation}
%
where $f_i(j;\vec{\theta})$ denotes the probability mass function of $X_i$
conditional on model $\vec{\theta}$, evaluated at integer $j$,
%
\begin{equation}
  \label{eq:nbinom-pmf}
  f_i(j;\vec{\theta})
  = \frac{\Gamma(k + j)}{j! \Gamma(k)} \bigg(\frac{k}{k + \mu_i(\vec{\xi})}\bigg)^k \bigg(\frac{\mu_i(\vec{\xi})}{k + \mu_i(\vec{\xi})}\bigg)^j\,\qquad j = 0,1,\ldots\,.
\end{equation}
%
Substituting \eqref{eq:nbinom-pmf} in \eqref{eq:nbinom} and taking the
logarithm of both sides yields the negative log likelihood
%
\begin{equation}
  \label{eq:nll}
  \begin{aligned}
    -\ell(\vec{\theta}|\vec{x})
    &= -\log \mathcal{L}(\vec{\theta}|\vec{x}) \\
    &= -\sum_{i=1}^{n} \log f_i(x_i;\vec{\theta}) \\
    &= -\sum_{i=1}^{n} \log \bigg[ \frac{\Gamma(k + x_i)}{x_i! \Gamma(k)} \bigg( \frac{k}{k + \mu_i(\vec{\xi})} \bigg)^{k} \bigg( \frac{\mu_i(\vec{\xi})}{k + \mu_i(\vec{\xi})} \bigg)^{x_i} \bigg] \\
    &= -\sum_{i=1}^{n} \Big\{ \log \Gamma(k + x_i) - \sum_{j=1}^{x_i} \log j - \log \Gamma(k) \\
    &\hspace{1in} + k \log k - k \log (k + \mu_i(\vec{\xi})) + x_i \log \mu_i(\vec{\xi}) - x_i \log (k + \mu_i(\vec{\xi})) \Big\}\,.
  \end{aligned}
\end{equation}

\section{Likelihood gradient}

The derivative of $-\ell$ with respect to $\vec{\xi}$ is
%
\begin{equation}
  \begin{aligned}
    -\frac{\partial \ell}{\partial \vec{\xi}}(\vec{\xi},k|\vec{x})
    &= -\sum_{i=1}^{n} \bigg\{ -\frac{k \frac{\partial \mu_i}{\partial \vec{\xi}}(\vec{\xi})}{k + \mu_i(\vec{\xi})} + \frac{x_i \frac{\partial \mu_i}{\partial \vec{\xi}}(\vec{\xi})}{\mu_i(\vec{\xi})} - \frac{x_i \frac{\partial \mu_i}{\partial \vec{\xi}}(\vec{\xi})}{k + \mu_i(\vec{\xi})} \bigg\} \\
    &= -\sum_{i=1}^{n} \frac{\partial \mu_i}{\partial \vec{\xi}}(\vec{\xi}) \bigg\{\frac{x_i}{\mu_i(\vec{\xi})} - \frac{k + x_i}{k + \mu_i(\vec{\xi})} \bigg\} \\
    &= -\sum_{i=1}^{n}\frac{\partial \mu_i}{\partial \vec{\xi}}(\vec{\xi})\cdot\frac{x_i - \mu_i(\vec{\xi})}{\mu_i(\vec{\xi})}\cdot\frac{k}{k + \mu_i(\vec{\xi})}\,.
  \end{aligned}
\end{equation}
%
The derivative of $-\ell$ with respect to $k$ is
%
\begin{equation}
  -\frac{\partial \ell}{\partial k}(\vec{\xi},k|\vec{x})
  = -\sum_{i=1}^{n} \bigg\{ \psi^{(0)}(k + x_i) - \psi^{(0)}(k) + 1 + \log k - \log (k + \mu_i(\vec{\xi})) - \frac{k + x_i}{k + \mu_i(\vec{\xi})} \bigg\}\,,
\end{equation}
%
where $\psi^{(0)}$ denotes the polygamma function of order 0
(or digamma function).

\section{Likelihood Hessian}






\end{document}
